\documentclass[a4paper,12pt]{article}

\title{Laba}
\author{Мишина Аня}
\date{today}
\usepackage[T2A]{fontenc}
\usepackage[utf8]{inputenc}
\usepackage[english,russian]{babel}
\usepackage{amsmath,amsfonts,amssymb,amsthm,mathtools,bpchem}
\usepackage{fancyhdr}
\usepackage{indentfirst}
\usepackage{float}
\usepackage{etaremune}

\usepackage[margin=1in]{geometry}

\pagestyle{fancy}
\fancyhf{}
\rhead{\thepage}
\renewcommand{\headrulewidth}{0pt}

\thispagestyle{empty}


\begin{document}
\begin{titlepage}
\begin{center} 
 
\large Московский физико-технический институт\\
Факультет молекулярной и химической физики\\
\vspace{7cm}
\Large Лабораторная работа \\по курсу \\ Физические методы исследований\\
\textbf{\Huge <<Спектроскопия электронного парамагнитного резонанса >>}\\
\end{center} 

\vspace{5cm}
{\par 
	\raggedleft \large 
	\emph{Выполнили:}\\ 
	студенты 3 курса\\ 
	643 группы ФМХФ\\ 
	Зарубин Всеволод  \\ 
	Мишина Аня 
\par}
\begin{center}
\vfill \today
\end{center}
\end{titlepage}
\newpage
\setcounter{page}{2}

   
	\newpage
	\section{Аннотация}
	\par В данной работе мы исследовали влияние амплитуды высокочастотной модуляции на вид спектров ЭПР, путем регистрации спектра ЭПР ДФПГ при разных амплитудах модуляции, оценили максимально достижимую для данного прибора амплитуду модуляции постоянного магнитного поля. По экспериментальным данным определили константу спинового обмена $K_e$ в растворе $Mn^{2+}$, получили и проанализировали ЭПР спектр соли $Mn^{2+}$, определили константу сверхтонкого взаимодействия и объяснили образование сверхтонкой структуры для ионов $Mn^{2+}$.
	
	\section{Теоретические сведения}
	
	\subsection{Электронный парамагнитный резонанс (ЭПР)}
	
	Под действием внешнего магнитного  поля  магнитные  моменты электронов ориентируются  в зависимости  от    спинового магнитного  момента,  и  их энергетический уровень расщепляется на два. Поскольку в стационарном состоянии на более низких по энергии уровнях находится нескольно больше частиц $N_1>N_2$ а вероятности переходов $\omega_{12}=\omega_{21}$ итоге будут преобладать переходы снизу вверх $1\rightarrow 2$ т.е.  с поглощением энергии излучения. Подобное избрательное поглощение энергии системой парамагнитных частиц при опрееленном отношении напряженности постоянного магнитного поля к частоте радиоизлученияи получило название электронного парамагнитного резонанса.
	
	\subsection{Сверхтонкая структура спектров ЭПР}
	 
	Если кроме неспаренных электронов исследуемый парамагнитный образец содержит атомные ядра, обладающие собственными моментами($_1^H$,$_2^D$, $_14^N$ и т.д.), то за счет взаимодействия электронных и ядерных магнитных моментов возникает сверхтонкая структура спектра.

	\subsection{Ширина спектральной линии}
	
	При соударении неспаренных электронов возможно изменение направления спинового момента, а значит, время жизни электрона в данном состоянии уменьшится. Тогда по принципу неопределенности Гейзенберга сигнал ЭПР уширится:
	\begin{equation}
	\delta E \cdot \tau \sim \hbar <=> \delta \omega \cdot \tau \sim 1
	\end{equation}
Время жизни $\tau$ определяется релаксационными процессами. Существует 2 типа релаксаций, влияющих на ширину спектральных линий: продольная (спин-решеточная релаксация) и поперечная релаксация. При продольной релаксации энергия из спиновой подсистемы уходит в другие степени свободы, так называемую решетку. Скорость релаксации характеризуется временем продольной релаксации Т1, за которое система спинов теряет  часть энергии, полученной при поглощении. В ходе поперечной релаксации энергия подсистемы спинов не изменяется, но происходит разориентация вкладов в поперечную намагниченность от разных частей образца. Это достигается спин-спиновой релаксацией (при взаимодействии двух частиц идет изменение спиновых состояний каждой). Характерное для данного типа релаксации время Т2 – время поперечной релаксации. 

	\section{Описание установки}
	
	Спектрометры ЭПР включают в себя в качестве обязательных элементов следующие устройства:

	\begin{enumerate}
	\item Генератор электромагнитного излучения. В современных спектрометрах ЭПР чаще всего используется излечение трехсантиметрового диапазона СВЧ($\lambda$ = 3 см), которое соответствует частоте $\nu$ = $10^{10}$ Гц.
	\item Волноводы - полые металлические трубы, имеющие в сечение прямоугольную форму. Волноводы предназнаечны для передачи электромагнитного СВЧ-излучения от генератора к образцу и от образца к детектору мощности СВЧ. 
	\item Объемные резанаторы, внутри которых концентрируется энергия электромагнитного излучения. Резонатор  необходим  для  усиления  слабого  сигнала  ЭПР.  Сигнал  в резонаторе  усиливается  пропорционально  добротности. В центральной части резанатора, где имеется пучность переменного магнитного поля, помещается исследуемый образец. 
	\item Магнитная система, состоящая из постоянных самарий-кобальтовых магнитов  и дополнительного электромагнита, установленного на стальных полюсных наконечниках. Электромагниты предназначены для развертки статического магнтиного поля. 
	\item Детектор электромагнитного излучения.
	\item Электронный усилитель сигнала , выдаваемого детектором.
	\item Регистрирующее устройство, на которое подается сигнал ЭПР. 
	\end{enumerate}
	
	\begin{figure}[h!]
		\begin{center}
			\includegraphics[scale=0.5]{schema}
			\caption{Схема экспериментально установки ЭПР-спектрометра}
		\end{center}
	\end{figure}
	
	
	Электромагнитное излучение сверхвысокой частоты (СВЧ) от источника (диод Ганна) по волноводам поступает в объемный резонатор, содержащий исследуемый образец и помещенный между полюсами электромагнита NS. В условиях резонанса СВЧ излучение поглощается спиновой системой. Модулированное поглощением СВЧ излучение по волноводу поступает на детектор. После детектирования сигнал усиливается на усилителе и подается на регистрирующее устройство.   Для повышения чувствительности и разрешения спектрометра ЭПР используют высокочастотную (ВЧ) модуляцию (обычно 100 кГц - 1 МГц) внешнего магнитного поля, осуществляемого с помощью модуляционных катушек. 


\newpage
	
	\section{Экспериментальная часть}
	
	\subsection{Исследование влияния амплитуды высокочастотной модуляции на вид спектров ЭПР}
	
	Был зарегистрирован спетр ЭПР тестового образца - ДФГП $(C\simeq 0.001 M)$ при нескольких амплитудах модуляции магнитного поля. Приведем полученные спектры при 8ми значениях тока модуляции:
	
	\begin{figure}[H]
		\begin{center}
			\includegraphics[scale=0.1]{amperage}
			\caption{Спектр ЭПР ДФГП при различных амплитудах тока модуляции}
		\end{center}
	\end{figure}
		
	
	\section{ Построим зависимость поглощения энергии СВЧ-поля от
концентрации раствора}
	    
	\begin{figure}[H]
		\begin{center}
			\includegraphics[scale=0.3]{S-absorb}
			\caption{Зависимость поглощения энергии\\*
СВЧ поля от концентрации раствора}
		\end{center}
	\end{figure}
    
	Построим зависимость полуширины линии поглощения $\delta H$ от величины тока модуляции.
\newpage
	\begin{figure}[H]
		\begin{center}
			\includegraphics[scale=0.3]{DELTA_FOR_I}
			\caption{зависимость $\delta H$ от  тока модуляции}
		\end{center}
	\end{figure}
	
	Из первого рисунка видно, что при значениях тока больше 1,25 А интенсивность сигнала уменьшается, поэтому имеет смысл брать ток модуляции не более 1,25 А. В дальнейших опытах ток модуляции был равен  0,75 А.
	%логично, что второй график мы не просто так строили, значит из него необходимо найти максимальную амплитуду модуляции, из какой-то лабы это величина равна коэфф. b в уравнении прямой, у нас это значение слишком мальенькое

	\subsection{Исследование скорости спинового обмена в растворах и кристаллах}	
	
	Были зарегистрированы спектры ЭПР образцов-растворов соли $Mn^{2+}$ различных концентраций. Приведем полученнные спектры при 7ми значениях концентрации 	


	\begin{figure}[H]
		\begin{center}
			\includegraphics[scale=0.1]{concentration}
			\caption{ Спектр ЭПР ДФГП при различных концентрация С раствора соли $Mn_{2+}$}
		\end{center}
	\end{figure}
	
	
	
\newpage
	Построим зависимость полуширины линии поглощения $\delta H$ от концентрации С раствора соли $Mn_{2+}$.
	
	\begin{figure}[H]
		\begin{center}
			\includegraphics[scale=0.3]{DELTA_FOR_C}
			\caption{Зависимость $\delta H$ от концентрации С раствора соли $Mn_{2+}$}
		\end{center}
	\end{figure}
	
	По углу наклона графика определим константу спинового обмена по формуле,где $\gamma$ $\approx$ $17,6 \cdot 10^6$:
	
	\begin{equation}
	\delta H_e = \frac{1}{\gamma \tau_e} = K_e C \frac{1}{\gamma}
	\end{equation}
	
	%пересчитать наше значение, вероятно, график неправильный
	\begin{center}
	$K_e = 7 \cdot 10^8 \frac{1}{M \cdot C}$
	\end{center}
		
	
 	Были также зарегистрированы спектры ЭПР порошка соли $Mn^{2+}$ в сухой пробирке. Приведем данный спектр на одном рисунке с одним из спектров растворов соли.
 	
 	
 	\begin{figure}[H]
		\begin{center}
			\includegraphics[scale=0.1]{solids}
			\caption{Спектры ЭПР в кристалле и в растворе соли $Mn_{2+}$}
		\end{center}
	\end{figure}
	
	
	Отсутствие линий поглощения для порошка соли марганца на рисунке вызвано высокой концентрацией парамагнитных частиц. Из-за быстрого обмена линии уширяются и сливаются в одну.
	
	\subsection{Исследование сверхтонкой структуры ЭПР}
	
	%нужно вставить наш график с 6 пиками и из него получить величину а, равную расстоянию между соседними пиками
	 Для ядра $Mn^{2+}$ полный магнитный момент $J=5/2$. Условие резонанса будет выполняться для электронов $2J+1=6$, т.е. поизойдет шестикратное расщепление линии поглощения. Константу сверхтонкого взаимодествия определяется по расстояние между линиями сверхтонкой структуры. В нашем случае: $a_n = 1.4$ Э.
	 
	 Снимем спектр порошка мела. В нем наблюдаются 6 пиков большой интенсивности и по два пика меньшей интенсивности между ними. Наличие шести равноотстоящих линий указывает на расщепление ядром со спином 5/2, например, марганцем.
	 
	
	\begin{figure}[H]
		\begin{center}
			\includegraphics[scale=0.25]{MEL}
			\caption{Спектр ЭПР порошка мела\\* Спектры ЭПР соли Mn2+ в различных условиях}       
		\end{center}
	\end{figure}
	
	
 	\begin{figure}[H]
		\begin{center}
			\includegraphics[scale=0.25]{MEL_DIFF}
			\caption{Спектр ЭПР порошка мела}
		\end{center}
	\end{figure}
	 
	 
	\subsection{Исследование влияния уровня диэлектрических потерь на вид спектров ЭПР}
	Раствор соли марганца поместим в резонатор:
	\begin{etaremune}
	\item в капиляре;
	\item в пробирке, сохраняя ту же высоту столба жидкости, что и в (1)
	\item в пробирке, сохраняя то же количество парамагнитных центров, что и в (1), при таком же объеме, как в пункте (2).
	\end{etaremune}
	Снимем спектры и приведем их на одном графике:
	
	
 	\begin{figure}[H]
		\begin{center}
			\includegraphics[scale=0.3]{DIELECTRIC_LOSS}
			\caption{Спектры ЭПР соли $Mn_{2+}$ в различных условиях}
		\end{center}
	\end{figure}
	
	
	
	Сигнал растворов соли марганца в пробирке и капиляре существенно разный. Более сильно выраженный минимум сигнала в пробирке связан с искривленной формой контура поглощения. Поле в пробирке более неоднородное, потому что что диаметр пробирки больше диаметра капилляра. Увеличение неоднородности ведет к увеличению электрического поля, а значит, к увеличению диэлектрических потерь. При добавлении воды амплитуда спектра уменьшается, что связано с диэлектрическими потерями сигнала в воде, которые велики при больших частотах переменного поля. Также уменьшение амплитуды подтверждает предположение о пропорциональности амплитуды сигнала ЭПР и количества парамагнитных частиц, находящихся в объеме резонатора.
	
	
	\subsection{Исследование формы линии}
	
	Согласно рисунку 4, форма линии для низких концентраций лучше описывается лоренцевским контуром, а для высоких ($\sim1 M$)- гауссовым. Подобное раличие форм линий связано с количеством парамагнитных центров в растворе и, как следствие, скоростью релаксации, связанной со спиновым обменом.
	
	
	\section{Вывод}
	
	\begin{itemize}
	\item Скорость спинового обмена зависит только от константы $K_e$.
	\item Увеличение амплитуды модуляции уменьшает шумы на спектре, но увеличивает ширину пиков. Оптимальная величина тока модуляции была принята равной 0.75 А, максимальная
амплитуда модуляции – 1.8 А.
	\item В порошке соли марганца линии поглощения не наблюдаются из-за большой концентрации: столкновения частиц происходит часто, поэтому имеем быстрый обмен, из-за которого все линии уширяются и сливаются в одну.
	\item Была определена константа сверхтонкого взаимодействия для раствора 𝑀𝑛2+ (𝑎 = 1.4 ± 0.001 Э).
	\item Из полученного спектра порошка мела можно сделать вывод о содержании в нем ядер со спином 5/2.
	\item Константа спинового обмена приняла значение
𝐾𝑒 = (7±2)· 10^8 \frac{л}{моль·с}
	\item В разбавленном растворе сигнал имеет меньшую амплитуду, что связано с уменьшением концентрации парамагнитных частиц и диэлектрическими потерями.
	\item При высоких концентрациях форма линии приближается гауссовой,
	но при уменьшении концентрации форма линии больше лучше описывается лоренцевской кривой.
	\end{itemize}
\vspace{4cm}
\section{Литература}
	 Спектроскопия электронного парамагнитного резонанса: учеб.-метод.
пособие / сост.: Е.Н.Кукаев, А.Ю.Куксин, А.О.Тишкина – М.: МФТИ,
2016. – 36 с.

\newpage
\section{Приложение}
	Настройка модуляции, может здесь фишка значений
\begin{figure}[H]
		\begin{center}
			\includegraphics[scale=0.1]{test}
		\end{center}
	\end{figure}
	
	
\begin{figure}[H]
		\begin{center}
			\includegraphics[scale=0.6]{table_1}
		\end{center}
	\end{figure}
	
	
\vspace{3cm}
\begin{figure}[H]
		\begin{center}
			\includegraphics[scale=0.5]{table_2}
		\end{center}
	\end{figure}
\end{document}

